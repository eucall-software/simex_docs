\documentclass[12pt]{scrartcl}
\usepackage{graphicx}
\usepackage{multirow}
\usepackage[default]{opensans}
\usepackage{sfmath} % sans font also for math
 \usepackage[binary-units = true]{siunitx}

% defining the paper layout that no text overlaps with the header
\usepackage[
  top=35mm,
  headheight=25mm,
  headsep=3mm,
  bottom=30mm,
  left=25mm,
  right=25mm
]{geometry}

\usepackage{latexsym}
\usepackage[centertags]{amsmath}
\usepackage{amssymb}

\graphicspath{{figures/}}
% custom header and footpage
\usepackage{scrpage2}
\pagestyle{scrheadings} % you have to set the custom layout
% Head
\ihead{M4.3} % left head
%\chead{}
\ohead{\includegraphics[height=25mm]{figures/EUCALL.png}}
% Foot
\ifoot{\includegraphics[height=13.4mm]{figures/EU.png}} % left foot
\cfoot{%
  \begin{minipage}{100mm}%
    \begin{scriptsize}%
      \normalfont{This project has received funding from the}
      \textit{European Union’s Horizon 2020 research and innovation programme}
      \normalfont{under grant agreement No 654220.}
    \end{scriptsize}%
  \end{minipage}%
} % center foot
\ofoot{\thepage} % right foot

\usepackage{booktabs}

%%%%%%%%%%%%%%%%%%%%%%%%%%%%%%%%%%%%%%%%%%%%%%%
%   BIBLIOGRAPHY SETTINGS                     %
%%%%%%%%%%%%%%%%%%%%%%%%%%%%%%%%%%%%%%%%%%%%%%%
\usepackage[bibstyle=nature,sorting=none,=maxnames=1000,eprint=false,
defernumbers=true, backend=biber]{biblatex}
\usepackage{hyperref}

\renewcommand*\finalnamedelim{, and\addspace}
\DeclareNameAlias{sortname}{last-first}
\renewcommand{\newunitpunct}{, }

\AtEveryBibitem{%
  \clearfield{day}%
  \clearfield{month}%
  \clearfield{endday}%
  \clearfield{endmonth}%
  \clearfield{issn}%
  \clearfield{issue}%
}
%convert titles to hyperlinks using doi
\ExecuteBibliographyOptions{doi=false} \newbibmacro{string+doi}[1]{%
  \iffieldundef{doi}{#1}{\href{http://dx.doi.org/\thefield{doi}}{#1}}}
  \DeclareFieldFormat*{title}{\usebibmacro{string+doi}{\mkbibemph{#1}}}

\addbibresource{references.bib}
\addbibresource{aux.bib}
\addbibresource{contributions/HPL/sample.bib}
%%%%%%%%%%%%%%%%%%%%%%%%%%%%%%%%%%%%%%%%%%%%%%%
% END BIBLIOGRAPHY SETTINGS                   %
%%%%%%%%%%%%%%%%%%%%%%%%%%%%%%%%%%%%%%%%%%%%%%%


% sophisticated linking of references in the pdf and setting some options
\usepackage{url}                                                  % for correct typesettings of URLs
\usepackage{hyperref}                                             % for sophisticated linking of urls, dois, pictures, tables, etc.
\hypersetup{
    unicode=true,                                                 % non-Latin characters in Acrobat’s bookmarks
    pdftoolbar=true,                                              % show Acrobat’s toolbar?
    pdfmenubar=true,                                              % show Acrobat’s menu?
    pdffitwindow=false,                                           % window fit to page when opened
    pdftitle={M4.3: Simulations interoperable},                   % title
    pdfauthor={C. Fortmann-Grote},                                % author
    pdfsubject={EUCALL WP4 (SIMEX) Milestone 4.3},                % subject of the document
    pdfcreator={pdflatex},                                        % creator of the document
    pdfkeywords={EUCALL, SIMEX, simulations},                     % list of keywords
    pdfnewwindow=true,                                           % links in new PDF window
    colorlinks=true,                                             % false: boxed links; true: colored links
    linkcolor=blue,                                              % color of internal links (change box color with linkbordercolor)
    citecolor=blue,                                              % color of links to bibliography
    filecolor=blue,                                              % color of file links
    urlcolor=blue                                                % color of external links
}

% Zeilenabstand
\renewcommand{\baselinestretch}{1.2}


%%%%%%%%%%%%%%%%%%%%%%%%%%%%%%%%%%%%%%%%%%%%%%%
% END BIBLIOGRAPHY SETTINGS                   %
%%%%%%%%%%%%%%%%%%%%%%%%%%%%%%%%%%%%%%%%%%%%%%%

\begin{document}
\makeatletter
\begin{titlepage}
\thispagestyle{scrheadings}
\begin{center}
  $~$\\
  \vspace{2cm}
  \Huge{\textbf{WP 4 -- SIMEX\\[1cm]
    Milestone M4.3: Demonstration of interoperability of simulations%
  }}\\
  \vspace{2cm}
  \large{Carsten Fortmann-Grote, Alexander Andreev, Ashutosh Sharma, Richard
    Briggs, Jan-Philipp Burchert, Marco Garten, Axel Huebl, Alexander Grund,
  Thomas Kluge, Sergey Yakubov, Michael Bussmann, and Adrian P. Mancuso}
  \vspace{1cm}
  \date{\today}
\end{center}
\vfill%
\includegraphics[width=\textwidth]{figures/PartnerLogos_2017}
\normalfont
\end{titlepage}
\makeatother
%
\tableofcontents
%
\section{Summary}\label{sec:summary}
\textbf{Milestone M4.3} (as detailed in \textbf{Task 4.2.2}) of the SIMEX
workpackage in EUCALL is delivery of interoperable simulations
(\textbf{Deliverable D4.3}) with documented examples. We have chosen five
applications to demonstrate the interoperability of our simulation codesThe examples are
\begin{enumerate}
  \item Simulation of diffraction of coherent x--rays delivered by an X-ray
    Free--Electron Laser from high energy density matter created by short--pulse
    laser matter interaction (as detailed in Deliverable D4.1)
  \item Simulation of x--ray fine structure absorption spectroscopy (XAFS) from long--pulse laser
    shocked warm dense matter at the synchrotron light source ESRF (as detailed
    in Deliverable D4.2).
  \item Serial femtosecond crystallography from protein crystals
  \item X--ray Thomson Scattering from warm dense plasmas
  \item Small angle X--ray Scattering from proteins
\end{enumerate}


These examples demonstrate the functionality and interoperability of all
simulation tools as demanded in Task 4.2.2.
Table~\ref{tab:simulation_capabilities} summarizes the available simulation
modules, the interfaced backengine simulation codes, and the relevant sections
of this report and external references where example applications can be found.

\begin{table}
  \begin{center}
    \scriptsize
    \caption{Demonstrated simulation capabilities}
    \label{tab:simulation_capabilities}
  \begin{tabular}[ht]{|l|l|l|l|}
    \hline
      \textbf{Task} &
      \textbf{Simulation capability} &
      \textbf{Simulation code}     &
      \textbf{Example references}    \\
    \hline
    \hline
      \multirow{3}{*}{X--ray sources}
      & XFEL source                     & FAST/XPD              & \cite{EUCALL_SIMEX_M4.1,EUCALL_SIMEX_M4.2,Fortmann-Grote2017}     \\
      & XFEL source                     & Genesis/Ocelot        & \ref{sec:lwfa_source} \\
      & Synchrotron source              & ShadowOUI             & \ref{sec:xafs}       \\
    \hline
    \multirow{2}{*}{Beam propagation}
      & Coherent wave propagaton        & WPG/SRW               & \cite{EUCALL_SIMEX_M4.1,EUCALL_SIMEX_M4.2,Fortmann-Grote2017}     \\
      & X--ray tracing                  & ShadowOUI             & \ref{sec:xafs}       \\
    \hline
    \multirow{3}{*}{Photon--matter interaction}
      & Radiation damage to molecules   & XMDYN/XATOM           & \cite{EUCALL_SIMEX_M4.1,EUCALL_SIMEX_M4.2,Fortmann-Grote2017}     \\
      & Short--pulse optical laser      & PIConGPU              & \ref{sec:lwfa_source},\ref{sec:plasma_diffraction},\cite{EUCALL_SIMEX_D4.1}  \\
      & Long--pulse optical laser       & Esther Rad--hydro     & \ref{sec:xafs} \cite{Torchio2016}  \\
    \hline
    \multirow{5}{*}{Signal generation}
      & Nano-particle scattering        & singFEL               & \cite{EUCALL_SIMEX_M4.1,EUCALL_SIMEX_M4.2,Fortmann-Grote2017}     \\

      & Plasma diffraction              & paraTAXIS             &
      \ref{sec:plasma_diffraction},\cite{EUCALL_SIMEX_D4.1,Kluge2016,
      Garten2017.zenodo.885033}  \\
      & XAFS                            & FEFF                  & \ref{sec:xafs}
      \cite{EUCALL_SIMEX_D4.2,Torchio2016,Harmand2015,Mazevet2014}  \\
      & Plasma XRTS                     & XRTScode              & \ref{sec:xrts} \cite{Fortmann2009d}               \\
      & Nano--crystallography           & CrystFEL              & \ref{sec:protein_sfx}         \\
    \hline
  \end{tabular}
  \end{center}
\end{table}

These codes are interfaced through the simulation platform ``simex\_platform'', which,
on the one side defines a python library to facilitate setup
and execution in a canonical way, and, on the other side, defines data interfaces
and formats to facilitate the passing of simulation data generated from one
module to the next.

\section{Coherent diffraction from high energy density matter}\label{sec:plasma_diffraction}
The interaction of \gls{uhi} lasers with solid matter at laser
pulse durations of few ten to hundred femtoseconds opens up the possibility to
study transient, non-equilibrium high energy density plasma processes on time
scales close to that of atomic processes with \glspl{xfel} with
nanometer resolution \cite{Kluge2016}, see also SIMEX Deliverable D4.1
\cite{EUCALL_SIMEX_D4.1}.

\begin{figure}
\centering
  \includegraphics[width=.95\linewidth]{figures/scattering_geometry_v4.png}
\caption{
  Scattering geometry with \gls{xfel} pulse perpendicular to optical
  \gls{uhi} laser. The
target is a silicon foil with a grating surface. As the \gls{xfel} pulse traverses the
target, the electron density changes due to the \gls{uhi} laser interacting with the
grating which dissolves over time. Delay times of the \gls{xfel} pulse maximum are
given with respect to the time when the optical laser pulse maximum hits the
target surface.  The area illuminated by the \gls{xfel} pulse is
$2\lambda_\mathrm{opt} \times 2\lambda_\mathrm{opt}$, with the corresponding
\gls{saxs} image, assuming $\SI{3}{\micro\metre}$ target depth, seen in Fig.
\ref{fig:scattering}.  }
  \label{fig:density}
\end{figure}

\begin{figure}
\centering
  \includegraphics[width=.85\linewidth]{figures/simex_workflow_v2.png}
\caption{
SIMEX workflow connecting the simulation codes via data exchange in \textit{openPMD} standard.
}
  \label{fig:workflows}
\end{figure}

Thus, radiation transport calculations must take these time and length scales
into account. We here introduce the example of softening and expansion of a
grating structure, see Fig. \ref{fig:density},  that is irradiated by an \gls{uhi}
optical laser pulse to illustrate the possibilities of \textit{ParaTAXIS}, a tool
developed within WP4 that resolves radiation transport on the single photon
level. \textit{ParaTAXIS} is fully integrated into the \textit{simex\_platform} tool chain via
\textit{openPMD} \cite{Huebl2017} in- and output, see Fig. \ref{fig:workflows}.

We study a $\tau_\mathrm{opt} = \SI{83}{\fs}$ \gls{fwhm} duration, wavelength $\lambda_\mathrm{opt} =
\SI{0.8}{\micro\metre}$ laser pulse impinging on a silicon foil under oblique incidence
that drives the grating into a heated plasma state. An x--ray pulse of
$\tau_\mathrm{XFEL} = \SI{10}{\fs}$ (\gls{fwhm}) with photon energy
$E_\mathrm{XFEL} = \SI{8.4}{\kilo\electronvolt}$ probes the surface grating structure
perpendicularly to the optical laser propagation with a delay relative to its
pulse maximum arriving at the target surface. With growing delay we expect the
scattering maxima of the grating to vanish as its edges soften and the plasma
expands into the vacuum.

The time structure of the \gls{xfel} pulse, the evolution of the target while the
pulse probes it and effects like multiple-scattering smear out the scattering
maxima. All effects are taken into account by \textit{ParaTAXIS}. For SIMEX milestone 4.3
we present the interoperability of \gls{xfel} and \gls{uhi} laser pulse generation,
interaction with the target and generation of a small-angle x--ray scattering
(\gls{saxs}) image.

A particle-in-cell (PIC) simulation (milestone 4.2.2.7) provides the time
evolution of the electron density on which the \gls{xfel} photons are scattered. We
assume invariance of the target in propagation direction and simulate the \gls{xfel}
pulse with \num{e12} photons for which the target is optically thin (milestone
4.2.2.9), see left part of Fig. \ref{fig:scattering}. We further demonstrate
scattering on an optically thick setup (milestone 4.2.2.10) by simulating
resonant scattering on the ions of the target. The ion density follows the
electron density as expected for plasma expansion into vacuum\cite{Mora2003},
see right part of Fig. \ref{fig:scattering}.

As expected, the signal of the optically thick target is washed out due to
higher scattering probability. In both, the optically thin and the optically
thick case, the
\gls{saxs} pattern well resolves the nanometer-scale grating depth and period, taking
into account the target evolution during the interaction time with the laser
pulse.

All density data from the PIC simulation as well as the \gls{saxs} patterns are
published on Zenodo together with the data format documentation
\cite{Garten2017.zenodo.885033}.

\begin{figure}
\centering
  \includegraphics[width=.99\linewidth]{figures/scattering_images_v2.png}
\caption{
\textbf{Left:} \textit{ParaTAXIS} \gls{saxs} image for the optically thin target at
$\SI{1.4}{\metre}$ distance from the target, detector pixel size $a_D =
\SI{13.5}{\micro\metre}$, X-ray wavelength $\lambda_\mathrm{\gls{xfel}} =
\SI{1.47}{\angstrom}$ and
\num{e12} photons in the illuminated area. The vertical separation of scattering
lines corresponds to the grating period of \SI{200}{\nano\metre}, the horizontal to
the grating depth of \SI{100}{\nano\metre}.
\textbf{Right:} \textit{ParaTAXIS} \gls{saxs} image for the optically thick target. Here, the
scattering cross section was increased by a factor of \num{1000} to account for
resonant scattering at the ion density. All other parameters remain the same.  }
  \label{fig:scattering}
\end{figure}





\section{X--ray fine structure absorption spectroscopy\label{sec:xafs}}
%
We have executed the simulations outlined in Deliverable
4.2\cite{EUCALL_SIMEX_D4.2}, describing an XAFS experiment on shock compressed
solid matter. There are three areas of simulation that are considered to describe the
whole source-to-end experiment: X-ray source \& ray tracing, long pulse photon-matter
interaction, and X-ray absorption modelling. Currently, the simulations of X-ray
source/X-ray tracing and XAS modelling are standalone simulations that provide
important information for the long pulse hydrocode simulations. Information
such as X-ray beam size on sample, will influence the laser conditions that can
be accepted for hydrocode simulation (the laser spot must be larger than the
X-ray spot size).

In this example, we run through the requirements that are necessary for
simulation (fitting) of EXAFS data relevant to a sample experiment obtain signal
from a \SI{5}{\micro\metre} thick Fe foil. The simulation will be carried out for
ambient conditions Fe foil; a future enhancement of SIMEX will take P-T
conditions from hydrocode simulations before performing high-pressure
XANES/EXAFS calculations.

To validate the EXAFS simulations, we compare the simulated spectrum with raw
data obtained from an X-ray absorption beamline at the ESRF synchrotron. The
normalized XAS spectrum for a \SI{5}{\micro\metre} thick Fe foil is shown in figure 1.
\begin{figure}
  \includegraphics[width=4.822in,height=2.528in]{figures/Task42210-img001.png}
  \caption{%
    X-ray absorption spectra collected at BM23 (EXAFS beamline, ESRF) of
    a \SI{5}{\micro\metre} thick iron foil.%
  }
  \label{fig:xafs_fig1}
\end{figure}
Calculations of XAFS spectra are performed using the FEFF package. FEFF uses a
single input file to select which modules should be run inside the program and
what parameters should be used. The material of interest is contained within
this input file based on its crystallographic parameters and atomic positions.
The ATHENA program is able to combine crystallographic input files (.cif format)
for a chosen material into the FEFF .inp format. The .cif files can typically be
found from most crystallographic database websites or can be manually created
using gui programs such as VESTA. FEFF is then run to calculate the paths
between atoms and is then saved / exported for use by other third party programs
(such as ARTEMIS) to compare with actual data.

\subsection{Artemis User Guide} A more comprehensive user guide for running ARTEMIS
/ FEFF can be found at
\href{http://bruceravel.github.io/demeter/artug/index.html}{http://bruceravel.github.io/demeter/artug/index.html}.

\begin{figure}
  \includegraphics[width=4.3063in,height=4.6465in]{figures/Task42210-img002.png}
  \caption{%
    Screenshot of the output data collected in the ATOMS software after
    running Feff simulation of Fe at ambient conditions.
  }
  \label{fig:xafs_fig2}
\end{figure}

\begin{figure}
  \includegraphics[width=4.852in,height=2.9173in]{figures/Task42210-img003.png}
  \caption{%
    Fitting of the first Fe shell from FEFF calculations (red) to Fe EXAFS
    data collected on BM23 beamline, ESRF (blue)
  }
  \label{fig:xafs_fig3}
\end{figure}

\subsection{Simulating XANES at shock conditions}
Shock compression experiments on Fe have previously been carried out at the ESRF
\cite{Torchio2016}. In that study,
simulations of the XANES at shock conditions were carried out using the ABINIT
code and are shown below in figure 4.

\begin{figure}
  \includegraphics[width=4.3063in,height=3.5945in]{figures/Task42210-img004.png}
  \caption{%
    Comparisons of the absorption edge of Fe between experiments (left
    panels) and ab-initio calculations (right panels) at
    \SIlist{120;150}{\giga\pascal}. Figure
  taken from Ref.~\cite{Torchio2016}.
  }
  \label{fig:xafs_fig4}
\end{figure}


\href{https://www.github.com/eucall-software/simex_platform/wiki/Esther-Hydrocode-Tutorial}{A
documented example workflow} is provided to demonstrate the usability of the
radiation--hydrodynamics simulation capabilities in \textit{simex\_platform}. It
shows how to optimize the target geometry (i.e. the thickness of the ablator
material) to maximize the data output, making use of the openPMD metadata
standard to facilitate transferrability among involved simulation codes and
data analysis and visualisation tools.

\section{Plasma X--ray Thomson Scattering\label{sec:xrts}}
TODO CFG
%
\section{Coherent diffraction from protein nano--crystals\label{sec:protein_sfx}}
In an
\href{https://github.com/eucall-software/simex_platform/wiki/Tutorial-on-nano-crystal-diffraction}{online
tutorial}, we demonstrate the usage of \textit{simex\_platform} to simulate
coherent diffraction of XFEL photons from nano--metre scale crystalline samples.
As in the other XFEL applications (see Sec.~\ref{sec:plasma_diffraction} and
Sec.~\ref{sec:xrts} as well as the first demonstration example in Milestone
M4.2 \cite{EUCALL_SIMEX_M4.2}), the source wavefront is queried from the
\href{https://in.xfel.eu/xpd}{XFEL Pulses Database XPD} and propagated to the
sample interaction point in the focus of the SPB--SFX instrument by means of
the coherent wavefront propagation code library WPG. Owing to the well defined
data interfaces and file format adaptors in \textit{simex\_platform} passing the
wavefront data to the crystal diffraction code is straightforward. We employ the
code \textit{pattern\_sim} which is part of the crystallography software suite
\textit{CrystFEL} \cite{White2012}, available as open source from the
\href{https://www.desy.de/~twhite/crystfel}{CrystFEL website}. The
corresponding \textit{Calculator} in \textit{simex\_platform} is the
\href{https://eucall-software.github.io/simex_platform/#SimEx.Calculators.CrystFELPhotonDiffractor.CrystFELPhotonDiffractor}{\textit{CrystFELPhotonDiffractor}}.
The \textit{CrystFELPhotonDiffractor} extracts the mean photon energy, the energy
spectrum, the beam divergence, the beam diameter, and other beam characteristics
from the wavefront data. The sample must be specified by giving a PDB code along
with information about the size of the nano--crystal, e.g. the extension in x,y,
and z directions. By default, the sample geometry is rotated in space via a
randomly chosen rotation operator to mimic the unknown orientation of the sample
in the experiment. Each simulated pattern is stored in a separate hdf5 file.
After the calculation, one master hdf5 file is generated which links to the
individual patterns and which has the same hierarchy as output generated e.g. by
the
\href{https://eucall-software.github.io/simex_platform/#SimEx.Calculators.SingFELPhotonDiffractor.SingFELPhotonDiffractor}{\textit{SingFELPhotonDiffractor}}
for single particle coherent diffraction. This in turn ensures that the same
analysis tools (e.g.
\href{https://eucall-software.github.io/simex_platform/#SimEx.Analysis.DiffractionAnalysis.DiffractionAnalysis}{\textit{DiffractionAnalysis}})
can be applied to crystal diffraction data and to single partice diffraction
data.

%
\section{Laser--wakefield driven FEL source}\label{sec:lwfa_source}
% This file was converted to LaTeX by Writer2LaTeX ver. 1.4
% see http://writer2latex.sourceforge.net for more info
\documentclass{article}
\usepackage[latin1]{inputenc}
\usepackage[T3,T1]{fontenc}
\usepackage[english]{babel}
\usepackage[noenc]{tipa}
\usepackage{tipx}
\usepackage[geometry,weather,misc,clock]{ifsym}
\usepackage{pifont}
\usepackage{eurosym}
\usepackage{amsmath}
\usepackage{wasysym}
\usepackage{amssymb,amsfonts,textcomp}
\usepackage{array}
\usepackage{supertabular}
\usepackage{hhline}
\usepackage[pdftex]{graphicx}
\makeatletter
\newcommand\arraybslash{\let\\\@arraycr}
\makeatother
\setlength\tabcolsep{1mm}
\renewcommand\arraystretch{1.3}
\title{Conference title, upper and lower case, bolded, 18 point type, centered}
\author{Carsten Fortmann-Grote}
\date{2017-09-16}
\begin{document}
EUCALL-SIMEX: Development for Laser Plasma Accelerator driven Free Electron
Laser source 

A.Sharma1 Axel Huebl2, Carsten Fortmann-Grote3, and Michael Bussmann2

1ELI-LAPS, Szeged, Hungary.

2HZDR, Dresden, Germany.

3XFEL, Hamburg, Germany.


\bigskip


\bigskip


\bigskip

In proposed SIMEX development project, we focus to investigate the possible
route for experimental realization of laser plasma accelerator based coherent
light source (FELs)1. Since the conventional accelerator has the limit of the
gradient due to the structure surface field limit, the best gradient achieved
now is less than 100MeV/m. Laser plasma accelerator (LPA)2-4 will be the next
generation accelerator facilities with the ultrahigh gradient which will reach
100GeV/m. Due to the ultrahigh gradient of LPA, the accelerator facility will
be miniaturized greatly. 

Here we are employing an extension of SIMEX platform to enable simulation of
LWFA based coherent light sources (FELs) by coupling a particle-in-cell (PIC)
code (PIConGPU)5 that describes the LWFA to a FEL simulation code (GENESIS)6.


\bigskip

Proposed Setup for SIMEX:

Our proposed setup is shown in Fig.~1; starting from electron beam generation
from a laser plasma source to the generation of femtosecond and/or attosecond
EUV/XUV pulse from the radiation undulator at endpoint. In this scheme an
intense. 


\bigskip


\includegraphics[width=5.4165in,height=2.7374in]{lwfafel-img/lwfafel-img001.png}


Fig. 1. Scheme of the proposed setup


\bigskip

(TW/PW power) laser is focused on gas jet/ gas filled capillary for producing a
relativistic electron beam. Initial energy spread of the electron beam of a
LPAs (1-5\%) is typically much larger than that of a LINAC (about 0.05 \%), so
reduction of the slice energy spread is necessary. The electron beam is sent
through a modulator undulator (MU) together with a TW-power laser beam, where
the interaction between the electrons, the magnetic field of the undulator and
the electromagnetic field of the laser introduces a periodic energy modulation
of the electrons. This energy modulation leads to the formation of nanobunches
(ultrashort electron layers). The nanobunched electron beam then passes through
a radiator undulator (RU) consisting of a single or a few periods and creates
EUV/XUV pulses.


\bigskip

We proceed in this SIMEX development for LPA-FEL following two parallel path:


\bigskip

\begin{enumerate}
\item Development of SIMEX setup for LPA-FEL by combining PIConGPU and GENESIS,
\item Utilisation of experimentally achieved high quality electron beam as an
input to FEL simulation code GPT/GENESIS, for investigation of coherent
radiation generation.
\end{enumerate}

\bigskip

Since February, we investigated both case as listed below:


\bigskip

\begin{enumerate}
\item Development of SIMEX setup for LPA-FEL (with Carsten and HZDR team)
\end{enumerate}

\bigskip


\includegraphics[width=5.9425in,height=4.1882in]{lwfafel-img/lwfafel-img002.png}


Figure 2: Layout for Simulation Setup


\bigskip

\begin{enumerate}
\item Laser-plasma accelerator based single-cycle attosecond undulator source
(in collaboration with T. Zoltan and Prof. Hebling, Pecs University Hungary)
\end{enumerate}

\bigskip

Example Results: Complete details on the following results can be accessed at
\ \ https://arxiv.org/pdf/1708.09384.pdf

\begin{flushleft}
\tablefirsthead{}
\tablehead{}
\tabletail{}
\tablelasttail{}
\begin{supertabular}{m{3.2177598in}m{3.2747598in}}
\centering{\selectlanguage{english} 
\includegraphics[width=2.7016in,height=2.1583in]{lwfafel-img/lwfafel-img003.jpg}
(a)} &
\centering\arraybslash{\selectlanguage{english} 
\includegraphics[width=3.0083in,height=2.352in]{lwfafel-img/lwfafel-img004.jpg}
(b)}\\
\multicolumn{2}{m{6.5712595in}}{{\selectlanguage{english} Fig. 3 - GPT
Simulation Results: (a) CEP-controlled EUV waveforms (b) and its spatial beam
profile. }}\\
\end{supertabular}
\end{flushleft}
Figure shown here, \ displays the simulated waveform of the generated attosecond
pulse and the beam profile at 60 nm radiation wavelength. The location of the
waveform in fig (a) is shown in figure (b) marked with a symbol x. In other
wavelengths the shape of the attosecond pulses are nearly identical with the
shape shown in fig (a) so these pulses are CEP controlled too.


\bigskip


\bigskip

Acknowledgements

This project has received funding from the European Union Horizon 2020 research
and innovation programme under grant agreement No 654220.\ \ 

\begin{figure}
\centering
\begin{minipage}{0.3827in}
a)
\end{minipage}
\end{figure}
\begin{figure}
\centering
\begin{minipage}{0.3827in}
b)
\end{minipage}
\end{figure}
\begin{flushleft}
\tablefirsthead{}
\tablehead{}
\tabletail{}
\tablelasttail{}
\begin{supertabular}{ll}
~
 & ~
\\
\multicolumn{2}{l}{~
}\\
\end{supertabular}
\end{flushleft}

\bigskip


\bigskip


\bigskip

References 

[1] P. Emma, K. Bane, M. Cornacchia, Z. Huang, H. Schlarb, G. Stupakov, and D.
Walz, Phys. Rev. Lett. 92, 074801 (2004)

[2] W. P. Leemans et al, Nat. Phys. 2, 696 (2006)

[3] E. Esarey et al, Rev. Mod. Phys. 81, 1229 (2009)

[4] T. Tajima and J. M. Dawson, Phys. Rev. Lett. 43, 267 (1979).

[5] ~M. Bussmann et al,~Proceedings SC13: International Conference for High
Performance Computing, Networking, Storage and Analysis~\textbf{5-1}, 2013.

[6] GPT: online at http://www.pulsar.nl/gpt

[7] GENESIS: online at http://genesis.web.psi.ch/
\end{document}


%Simulations of compact coherent x--ray sources based on the mechanism of
%electron laser wakefield acceleration (LWFA) couple particle--in--cell
%simulations to an FEL code. In our case, we use the PIC code \textit{PIConGPU}
%and the Genesis FEL code. \textit{PIConGPU} writes the particle and field data
%into a hdf5 file using the openPMD \cite{Huebl2017} metadata standard. A file format conversion utility, which is part of
%\textit{simex\_platform} converts the PIC output to an electron distribution
%file (extension .dist) readable by \textit{Genesis}.

\printbibliography[notkeyword=report, notkeyword=zenodo, title={Journal articles}]
%
\printbibliography[keyword=eucall, keyword=report, title={EUCALL Reports}]
%
\printbibliography[keyword=zenodo, title={EUCALL Data Repository Depositions}]

\end{document}



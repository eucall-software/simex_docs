\documentclass[12pt]{scrartcl}
\usepackage{graphicx}
\usepackage{multirow}
\usepackage[default]{opensans}
\usepackage{sfmath} % sans font also for math
 \usepackage[binary-units = true]{siunitx}

% defining the paper layout that no text overlaps with the header
\usepackage[
  top=35mm,
  headheight=25mm,
  headsep=3mm,
  bottom=30mm,
  left=25mm,
  right=25mm
]{geometry}

\usepackage{latexsym}
\usepackage[centertags]{amsmath}
\usepackage{amssymb}

\graphicspath{{figures/}}
% custom header and footpage
\usepackage{scrpage2}
\pagestyle{scrheadings} % you have to set the custom layout
% Head
\ihead{M4.3} % left head
%\chead{}
\ohead{\includegraphics[height=25mm]{figures/EUCALL.png}}
% Foot
\ifoot{\includegraphics[height=13.4mm]{figures/EU.png}} % left foot
\cfoot{%
  \begin{minipage}{100mm}%
    \begin{scriptsize}%
      \normalfont{This project has received funding from the}
      \textit{European Union’s Horizon 2020 research and innovation programme}
      \normalfont{under grant agreement No 654220.}
    \end{scriptsize}%
  \end{minipage}%
} % center foot
\ofoot{\thepage} % right foot

\usepackage{booktabs}

%%%%%%%%%%%%%%%%%%%%%%%%%%%%%%%%%%%%%%%%%%%%%%%
%   BIBLIOGRAPHY SETTINGS                     %
%%%%%%%%%%%%%%%%%%%%%%%%%%%%%%%%%%%%%%%%%%%%%%%
\usepackage[bibstyle=nature,sorting=none,=maxnames=1000,eprint=false,
defernumbers=true, backend=biber]{biblatex}
\usepackage{hyperref}

\renewcommand*\finalnamedelim{, and\addspace}
\DeclareNameAlias{sortname}{last-first}
\renewcommand{\newunitpunct}{, }

\AtEveryBibitem{%
  \clearfield{day}%
  \clearfield{month}%
  \clearfield{endday}%
  \clearfield{endmonth}%
  \clearfield{issn}%
  \clearfield{issue}%
}
%convert titles to hyperlinks using doi
\ExecuteBibliographyOptions{doi=false} \newbibmacro{string+doi}[1]{%
  \iffieldundef{doi}{#1}{\href{http://dx.doi.org/\thefield{doi}}{#1}}}
  \DeclareFieldFormat*{title}{\usebibmacro{string+doi}{\mkbibemph{#1}}}

\addbibresource{aux.bib}
\addbibresource{references.bib}
\addbibresource{contributions/HPL/sample.bib}
%%%%%%%%%%%%%%%%%%%%%%%%%%%%%%%%%%%%%%%%%%%%%%%
% END BIBLIOGRAPHY SETTINGS                   %
%%%%%%%%%%%%%%%%%%%%%%%%%%%%%%%%%%%%%%%%%%%%%%%


% sophisticated linking of references in the pdf and setting some options
\usepackage{url}                                                  % for correct typesettings of URLs
\usepackage{hyperref}                                             % for sophisticated linking of urls, dois, pictures, tables, etc.
\hypersetup{
    unicode=true,                                                 % non-Latin characters in Acrobat’s bookmarks
    pdftoolbar=true,                                              % show Acrobat’s toolbar?
    pdfmenubar=true,                                              % show Acrobat’s menu?
    pdffitwindow=false,                                           % window fit to page when opened
    pdftitle={M4.3: Simulations interoperable},                   % title
    pdfauthor={C. Fortmann-Grote},                                % author
    pdfsubject={EUCALL WP4 (SIMEX) Milestone 4.3},                % subject of the document
    pdfcreator={pdflatex},                                        % creator of the document
    pdfkeywords={EUCALL, SIMEX, simulations},                     % list of keywords
    pdfnewwindow=true,                                           % links in new PDF window
    colorlinks=true,                                             % false: boxed links; true: colored links
    linkcolor=blue,                                              % color of internal links (change box color with linkbordercolor)
    citecolor=blue,                                              % color of links to bibliography
    filecolor=blue,                                              % color of file links
    urlcolor=blue                                                % color of external links
}

% Zeilenabstand
\renewcommand{\baselinestretch}{1.2}


%%%%%%%%%%%%%%%%%%%%%%%%%%%%%%%%%%%%%%%%%%%%%%%
% END BIBLIOGRAPHY SETTINGS                   %
%%%%%%%%%%%%%%%%%%%%%%%%%%%%%%%%%%%%%%%%%%%%%%%

\begin{document}
\makeatletter
\begin{titlepage}
\thispagestyle{scrheadings}
\begin{center}
  $~$\\
  \vspace{2cm}
  \Huge{\textbf{WP 4 -- SIMEX\\[1cm]
    Milestone M4.3: Demonstration of interoperability of simulations%
  }}\\
  \vspace{2cm}
  \large{Carsten Fortmann-Grote, Alexander Andreev, Ashutosh Sharma, Richard
    Briggs, Jan-Philipp Burchert, Marco Garten, Axel Huebl, Alexander Grund,
  Thomas Kluge, Sergey Yakubov, Michael Bussmann, and Adrian P. Mancuso}
  \vspace{1cm}
  \date{\today}
\end{center}
\vfill%
\includegraphics[width=\textwidth]{figures/PartnerLogos_2017}
\normalfont
\end{titlepage}
\makeatother
%
\tableofcontents
%
\section{Summary}\label{sec:summary}
\textbf{Milestone M4.3} (as detailed in \textbf{Task 4.2.2}) of the SIMEX
workpackage in EUCALL is delivery of interoperable simulations
(\textbf{Deliverable D4.3}) with documented examples. We have chosen NN
applications to demonstrate the interoperability of our simulation codes. These
codes are interfaced through the simulation platform ``simex\_platform'', which,
on the one side defines python APIs through which the simulations can be set up
and executed in a canonical way, and, on the other side, defines data interfaces
and formats to facilitate the passing of simulation data generated from a given
code A (data source) to another simulation code B (data sink), which processes
this input data and produces another set of output data for a simulation code C,
and so on.

The examples are
\begin{enumerate}
  \item Simulation of diffraction of coherent x--rays delivered by an X-ray
    Free--Electron Laser from high energy density matter created by short--pulse
    laser matter interaction (as detailed in Deliverable D4.1)
  \item Simulation of x--ray fine structure absorption spectroscopy (XAFS) from long--pulse laser
    shocked warm dense matter at the synchrotron light source ESRF (as detailed
    in Deliverable D4.2).
  \item Serial femtosecond crystallography from protein crystals
  \item X--ray Thomson Scattering from warm dense plasmas
  \item Small angle X--ray Scattering from proteins
\end{enumerate}

These examples demonstrate the functionality and interoperability of all
simulation tools as demanded in Task 4.2.2. To ease the structuring and
sharing of work among participating institutes and collaborators, we introduced the following subtask enumeration

\begin{table}
  \begin{center}
    \scriptsize
    \caption{Demonstrated simulation capabilities}
    \label{tab:simulation_capabilities}
  \begin{tabular}[ht]{|l|l|l|l|}
    \hline
      \textbf{Task} &
      \textbf{Simulation capability} &
      \textbf{Simulation code}     &
      \textbf{Example reference}    \\
    \hline
    \hline
      \multirow{3}{*}{X--ray sources}
      & XFEL source                     & FAST/XPD              & \cite{EUCALL_SIMEX_M4.1,EUCALL_SIMEX_M4.2,Fortmann-Grote2017}     \\
      & XFEL source                     & Genesis/Ocelot        & \ref{sec:lwfa_source} \\
      & Synchrotron source              & ShadowOUI             & \ref{sec:xafs}       \\
    \hline
    \multirow{2}{*}{Beam propagation}
      & Coherent wave propagaton        & WPG/SRW               & \cite{EUCALL_SIMEX_M4.1,EUCALL_SIMEX_M4.2,Fortmann-Grote2017}     \\
      & X--ray tracing                  & ShadowOUI             & \ref{sec:xafs}       \\
    \hline
    \multirow{3}{*}{Photon--matter interaction}
      & Radiation damage to molecules   & XMDYN/XATOM           & \cite{EUCALL_SIMEX_M4.1,EUCALL_SIMEX_M4.2,Fortmann-Grote2017}     \\
      & Short--pulse optical laser      & PIConGPU              & \ref{sec:lwfa_source},\ref{sec:plasma_diffraction},\cite{EUCALL_SIMEX_D4.1}  \\
      & Long--pulse optical laser       & Esther Rad--hydro     & \ref{sec:xafs} \cite{Torchio2016}  \\
    \hline
    \multirow{5}{*}{Signal generation}
      & Nano-particle scattering        & singFEL               & \cite{EUCALL_SIMEX_M4.1,EUCALL_SIMEX_M4.2,Fortmann-Grote2017}     \\

      & Plasma diffraction              & paraTAXIS             &
      \ref{sec:plasma_diffraction},\cite{EUCALL_SIMEX_D4.1,Kluge2016, Garten2017}  \\
      & XAFS                            & FEFF                  & \ref{sec:xafs} \cite{EUCALL_SIMEX_D4.2,Torchio2016}  \\
      & Plasma XRTS                     & XRTScode              & \ref{sec:xrts} \cite{Fortmann2009d}               \\
      & Nano--crystallography           & CrystFEL              & \ref{sec:protein_sfx}         \\
    \hline
  \end{tabular}
  \end{center}
\end{table}

\section{Coherent diffraction from high energy density matter}\label{sec:plasma_diffraction}
The interaction of \gls{uhi} lasers with solid matter at laser
pulse durations of few ten to hundred femtoseconds opens up the possibility to
study transient, non-equilibrium high energy density plasma processes on time
scales close to that of atomic processes with \glspl{xfel} with
nanometer resolution \cite{Kluge2016}, see also SIMEX Deliverable D4.1
\cite{EUCALL_SIMEX_D4.1}.

\begin{figure}
\centering
  \includegraphics[width=.95\linewidth]{figures/scattering_geometry_v4.png}
\caption{
  Scattering geometry with \gls{xfel} pulse perpendicular to optical
  \gls{uhi} laser. The
target is a silicon foil with a grating surface. As the \gls{xfel} pulse traverses the
target, the electron density changes due to the \gls{uhi} laser interacting with the
grating which dissolves over time. Delay times of the \gls{xfel} pulse maximum are
given with respect to the time when the optical laser pulse maximum hits the
target surface.  The area illuminated by the \gls{xfel} pulse is
$2\lambda_\mathrm{opt} \times 2\lambda_\mathrm{opt}$, with the corresponding
\gls{saxs} image, assuming $\SI{3}{\micro\metre}$ target depth, seen in Fig.
\ref{fig:scattering}.  }
  \label{fig:density}
\end{figure}

\begin{figure}
\centering
  \includegraphics[width=.85\linewidth]{figures/simex_workflow_v2.png}
\caption{
SIMEX workflow connecting the simulation codes via data exchange in \textit{openPMD} standard.
}
  \label{fig:workflows}
\end{figure}

Thus, radiation transport calculations must take these time and length scales
into account. We here introduce the example of softening and expansion of a
grating structure, see Fig. \ref{fig:density},  that is irradiated by an \gls{uhi}
optical laser pulse to illustrate the possibilities of \textit{ParaTAXIS}, a tool
developed within WP4 that resolves radiation transport on the single photon
level. \textit{ParaTAXIS} is fully integrated into the \textit{simex\_platform} tool chain via
\textit{openPMD} \cite{Huebl2017} in- and output, see Fig. \ref{fig:workflows}.

We study a $\tau_\mathrm{opt} = \SI{83}{\fs}$ \gls{fwhm} duration, wavelength $\lambda_\mathrm{opt} =
\SI{0.8}{\micro\metre}$ laser pulse impinging on a silicon foil under oblique incidence
that drives the grating into a heated plasma state. An x--ray pulse of
$\tau_\mathrm{XFEL} = \SI{10}{\fs}$ (\gls{fwhm}) with photon energy
$E_\mathrm{XFEL} = \SI{8.4}{\kilo\electronvolt}$ probes the surface grating structure
perpendicularly to the optical laser propagation with a delay relative to its
pulse maximum arriving at the target surface. With growing delay we expect the
scattering maxima of the grating to vanish as its edges soften and the plasma
expands into the vacuum.

The time structure of the \gls{xfel} pulse, the evolution of the target while the
pulse probes it and effects like multiple-scattering smear out the scattering
maxima. All effects are taken into account by \textit{ParaTAXIS}. For SIMEX milestone 4.3
we present the interoperability of \gls{xfel} and \gls{uhi} laser pulse generation,
interaction with the target and generation of a small-angle x--ray scattering
(\gls{saxs}) image.

A particle-in-cell (PIC) simulation (milestone 4.2.2.7) provides the time
evolution of the electron density on which the \gls{xfel} photons are scattered. We
assume invariance of the target in propagation direction and simulate the \gls{xfel}
pulse with \num{e12} photons for which the target is optically thin (milestone
4.2.2.9), see left part of Fig. \ref{fig:scattering}. We further demonstrate
scattering on an optically thick setup (milestone 4.2.2.10) by simulating
resonant scattering on the ions of the target. The ion density follows the
electron density as expected for plasma expansion into vacuum\cite{Mora2003},
see right part of Fig. \ref{fig:scattering}.

As expected, the signal of the optically thick target is washed out due to
higher scattering probability. In both, the optically thin and the optically
thick case, the
\gls{saxs} pattern well resolves the nanometer-scale grating depth and period, taking
into account the target evolution during the interaction time with the laser
pulse.

All density data from the PIC simulation as well as the \gls{saxs} patterns are
published on Zenodo together with the data format documentation
\cite{Garten2017.zenodo.885033}.

\begin{figure}
\centering
  \includegraphics[width=.99\linewidth]{figures/scattering_images_v2.png}
\caption{
\textbf{Left:} \textit{ParaTAXIS} \gls{saxs} image for the optically thin target at
$\SI{1.4}{\metre}$ distance from the target, detector pixel size $a_D =
\SI{13.5}{\micro\metre}$, X-ray wavelength $\lambda_\mathrm{\gls{xfel}} =
\SI{1.47}{\angstrom}$ and
\num{e12} photons in the illuminated area. The vertical separation of scattering
lines corresponds to the grating period of \SI{200}{\nano\metre}, the horizontal to
the grating depth of \SI{100}{\nano\metre}.
\textbf{Right:} \textit{ParaTAXIS} \gls{saxs} image for the optically thick target. Here, the
scattering cross section was increased by a factor of \num{1000} to account for
resonant scattering at the ion density. All other parameters remain the same.  }
  \label{fig:scattering}
\end{figure}





\section{X--ray fine structure absorption spectroscopy\label{sec:xafs}}
%
We have executed the simulations outlined in Deliverable
4.2\cite{EUCALL_SIMEX_D4.2}, describing an XAFS experiment on shock compressed
solid matter.
The synchrotron X-ray source and propagation from the source to the target is
realized with the \href{http://ftp.esrf.eu/pub/scisoft/Oasys/readme.html}{Oasys}
software package developed by Luca Rebuffi and Manuel Sanchez del Rio
 using the x--ray tracing
capability via the code \textit{Shadow3}. Installation scripts and the wiki page for Oasys 1.0 can be
found at
\href{https://github.com/srio/oasys-installation-scripts/wiki}{https://github.com/srio/oasys-installation-scripts/wiki}.
\href{https://github.com/srio/ShadowOui-Tutorial}{Tutorials for
Oasys and ShadowOUI} cover the various steps in preparing, executing, and
analysing the raytracing simulation.

As part of preparation for the new High Power
Laser Facility (HPLF), that will be installed on the ID24 beamline at the ESRF,
the current energy dispersive X-ray absorption beamline has been simulated using
Oasys.
The ID24 beamline workflow for Oasys can be obtained from the EUCALL Data
Repository at Zenodo \cite{Briggs2017.zenodo.886451}.
\section{Plasma X--ray Thomson Scattering\label{sec:xrts}}
TODO CFG
%
\section{Coherent diffraction from protein nano--crystals\label{sec:protein_sfx}}
TODO CFG
%
\section{LWFA source}\label{sec:lwfa_source}
Simulations of compact coherent x--ray sources based on the mechanism of
electron laser wakefield acceleration (LWFA) couple particle--in--cell
simulations to an FEL code. In our case, we use the PIC code \textit{PIConGPU}
and the Genesis FEL code. \textit{PIConGPU} writes the particle and field data
into a hdf5 file using the openPMD \cite{Huebl2015} metadata standard. A file format conversion utility, which is part of
\textit{simex\_platform} converts the PIC output to an electron distribution
file (extension .dist) readable by \textit{Genesis}.

\printbibliography[notkeyword=report, notkeyword=zenodo, title={Journal articles}]
%
\printbibliography[keyword=eucall, keyword=report, title={EUCALL Reports}]
%
\printbibliography[keyword=zenodo, title={EUCALL Data Repository Depositions}]

\end{document}



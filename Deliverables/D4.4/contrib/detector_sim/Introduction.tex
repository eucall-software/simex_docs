\subsection{Detector simulation software}
For the purpose of detector simulations, the software packages \texttt{Geant4}
(\url{http://geant4.cern.ch/}) and \textit{X-CSIT}
(\url{https://git.xfel.eu/gitlab/karaboDevices/xcsit}) are utilized that have
already been successfully implemented in the data analysis and control framework
of the European XFEL \textit{karabo}
(\url{https://git.xfel.eu/gitlab/Karabo/Framework}). In contrast to
\textit{karabo}, where all the simulation is performed with devices which are
integrated into the framework processing pipe, \textit{SimEx} is a collection of
classes. Consequently,  installation as well as
maintenance and usage of new components are more convenient in \textit{SimEx}.


Another difference is the programming language:
In contrast to \textit{karabo} and \textit{X-CSIT} which are
written in C++, \textit{SimEx} is written in python. Since \textit{X-CSIT} is
the basis of the simulation also in this project, the interface defined needs to
be made accessible from python. To integrate it into \textit{SimEx} an
additional calculator needs to be written that utilizes the extended functions
from \textit{X-CSIT}. To achieve this, an interface between C++ written
\textit{X-CSIT} and python written \textit{SimEx} source code is designed and
implemented. This includes writing source code in C++ and python as well as
creating a build procedure with \textit{cmake}. Furthermore, an appropriate
documentation and similar coding style like the one used for other
\textit{SimEx} calculators is required.


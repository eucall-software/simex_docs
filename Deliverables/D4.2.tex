\documentclass[12pt]{scrartcl}
\usepackage{graphicx}
\usepackage[default]{opensans}
\usepackage{sfmath} % sans font also for math
 \usepackage[binary-units = true]{siunitx}

% defining the paper layout that no text overlaps with the header
\usepackage[
  top=35mm,
  headheight=25mm,
  headsep=3mm,
  bottom=30mm,
  left=25mm,
  right=25mm
]{geometry}

\usepackage{latexsym}
\usepackage[centertags]{amsmath}
\usepackage{amssymb}

% custom header and footpage
\usepackage{scrpage2}
\pagestyle{scrheadings} % you have to set the custom layout
\ihead{D4.1: Short Pulse Laser -- Matter Interaction} % left head
\ohead{\includegraphics[height=25mm]{EUCALL.png}}
\ifoot{\includegraphics[height=13.4mm]{EU.png}} % left foot
\cfoot{
  \begin{minipage}{100mm}
    \begin{scriptsize}
      \normalfont{This project has received funding from the}
      \textit{European Union’s Horizon 2020 research and innovation programme}
      \normalfont{under grant agreement No 654220.}
    \end{scriptsize}
  \end{minipage}
} % center foot
\ofoot{} % right foot
\chead{}

\usepackage{booktabs} % more eye candy for tables

% sophisticated linking of references in the pdf and setting some options
\usepackage{url}                                                  % for correct typesettings of URLs
\usepackage{hyperref}                                             % for sophisticated linking of urls, dois, pictures, tables, etc.
\hypersetup{
    unicode=true,                                                 % non-Latin characters in Acrobat’s bookmarks
    pdftoolbar=true,                                              % show Acrobat’s toolbar?
    pdfmenubar=true,                                              % show Acrobat’s menu?
    pdffitwindow=false,                                           % window fit to page when opened
    pdfstartview={FitH},                                          % fits the width of the page to the window
    pdftitle={D4.1: Short pulse laser-matter interaction simulations},                                        % title
    pdfauthor={C. Fortmann-Grote},                                           % author
    pdfsubject={EUCALL WP4 (SIMEX) deliverable 4.1},                             % subject of the document
    pdfcreator={pdflatex},                                         % creator of the document
    pdfkeywords={EUCALL, SIMEX, simulations, PIC, XFEL, scattering},                                         % list of keywords
    pdfnewwindow=true,                                            % links in new PDF window
    colorlinks=true,                                              % false: boxed links; true: colored links
    linkcolor=blue,                                                % color of internal links (change box color with linkbordercolor)
    citecolor=blue,                                                % color of links to bibliography
    filecolor=blue,                                               % color of file links
    urlcolor=blue                                                 % color of external links
}


% for fast reference
\newcommand{\fig}[1]{{figure\,\ref{#1}}}
\newcommand{\tab}[1]{{table\,\ref{#1}}}
\newcommand{\eq}[1]{{equation~(\ref{#1})}}


% Zeilenabstand
\renewcommand{\baselinestretch}{1.2}


%\renewcommand*\chapterpagestyle{scrheadings} % otherwise chapter start is plain ;-); works only for KOMA script; plain LaTeX works different

%%%%%%%%%%%%%%%%%%%%%%%%%%%%%%%%%%%%%%%%%%%%%%%
%   BIBLIOGRAPHY SETTINGS                     %
%%%%%%%%%%%%%%%%%%%%%%%%%%%%%%%%%%%%%%%%%%%%%%%
\usepackage[bibstyle=nature,sorting=none,=maxnames=1000,eprint=false,
defernumbers=true, backend=biber]{biblatex}
\usepackage{hyperref}

\renewcommand*\finalnamedelim{, and\addspace}
\DeclareNameAlias{sortname}{last-first}
\renewcommand{\newunitpunct}{, }

\AtEveryBibitem{%
  \clearfield{day}%
  \clearfield{month}%
  \clearfield{endday}%
  \clearfield{endmonth}%
  \clearfield{issn}%
  \clearfield{issue}%
}
%convert titles to hyperlinks using doi
\ExecuteBibliographyOptions{doi=false} \newbibmacro{string+doi}[1]{%
  \iffieldundef{doi}{#1}{\href{http://dx.doi.org/\thefield{doi}}{#1}}}
  \DeclareFieldFormat*{title}{\usebibmacro{string+doi}{\mkbibemph{#1}}}

%\addbibresource{/home/grotec/Documents/LiteratureDB/bibtex/jabref.bib}
\addbibresource{references.bib}
%%%%%%%%%%%%%%%%%%%%%%%%%%%%%%%%%%%%%%%%%%%%%%%
% END BIBLIOGRAPHY SETTINGS                   %
%%%%%%%%%%%%%%%%%%%%%%%%%%%%%%%%%%%%%%%%%%%%%%%

\begin{document}
\makeatletter
\begin{titlepage}
\thispagestyle{scrheadings}
\begin{center}
$~$\\
\vspace{2cm}
\Huge{\textbf{Design Report and Advanced Simulation Software for Long Pulse Laser -- Matter Interaction}}\\[5mm]
\vspace{2cm}
\large{
Carsten Fortmann-Grote, Alexander Andreev, Richard Briggs,\\ Michael Bussmann,
  Axel Huebl, Thomas Kluge,\\
 S. Pascarelli, Ashutosh Sharma, and Adrian P. Mancuso\\
 }
\vspace{1cm}
\@date
\end{center}
\vspace{5cm}
\includegraphics[width=\textwidth]{./PartnerLogos.pdf}
\normalfont
\end{titlepage}
\makeatother

\section{Introduction}
Long pulse laser systems are capable of reaching intensities of up to $\sim 10^{12}$ -- 10$^{15} \text{ W}/\text{cm}^2$ with pulse lengths of greater than a few nanoseconds. The pulse length and focal spot of the laser can depend heavily on the sample material and target package. Generally, focal spot sizes range between $\sim$ 100 $\mu$m up to 1 mm. With these specifications, the laser energies now required to reach the highest intensities are on the order of hundreds of joules.

Temporally shaped laser pulses interacting with overdense target material create a rapidly expanding plasma that
can generate a shockwave in the ablating material. This shock wave can travel through the target at several km/s
and compress the sample material to pressures reaching several hundred GPa. (For comparison, the Earth's core
pressure is $\sim$ 330 GPa). By using a ramp temporal pulse, where the laser intensity is slowly increased, the temperatures generated remain much cooler than during rapid shock compression and the solid state of matter can be studied up to several TPa (1 TPa = 1000 GPa).

High-power \marginpar{not "high energy" ?} laser facilities are now becoming increasingly in demand at new
generation x-ray light sources (X-FELs, 3rd generation synchrotron sources). The potential of long pulse laser experiments are already yielding extreme conditions of matter well beyond the reach of static high-pressure techniques and combinations with X-ray techniques provide enticing experiments at new extreme states of matter.

We design here simulations for a prototypical experiment combining a high-energy laser system and x-ray radiation
from a 3rd generation synchrotron. The laser shock-compresses a tailored solid density target and the compressed
matter is subsequently probed by x-rays. X-ray absorption spectroscopy (XAS) allows to monitor the compression and
to characterize the electronic and structural states in the target. Through variation of the delay between optical
laser pulse and x-ray probe pulse, time resolved data is obtained.

\begin{table}
  \centering
  \begin{tabular}{ll}
    \hline
    Photon energy & 5 keV - 28 keV \\
    Polarization & horizontal \\
    Focus size (fwhm) & 3 $\mu$m (5 keV) - 50 $\mu$m (28 keV)\\
    Pulse length & 100 ps\footnote{for shock compression experiments}\\
    Flux  & $1\times 10^{14}$ photons/s (7 keV) - $4\times 10^{13}$ photons/s (28 keV) \\
    \hline
  \end{tabular}
  \caption{Principal x-ray parameters for shock compression studies at ESRF
  (beamline ID24)}
  \label{tab:esrf_parameters}
\end{table}

The optical laser source, used at the ESRF for shock compression experiments in May/June 2016,  is described through a set of parameters:
\begin{table}
  \centering
  \begin{tabular}{ll}
    \hline
Wavelength & 1053 nm \\
Pulse duration & 4 ns - 15 ns \\
Intensity & $2\times 10^{13}$ W/cm$^2$ \\
Pulse energy & 30 J \\
Focus size (fhwm) & 100 $\mu\text{m}^{2}$ - 300 $\mu\text{m}^{2}$  \\
Temporal profile  & flat top \\
Spatial profile & Gaussian in x,y \\
    \hline
  \end{tabular}
  \caption{ESRF optical laser parameters for long pulse laser-matter
  simulations.}
  \label{tab:esrf_long_pulse}
\end{table}

A call for tender will be issued in 2016 for a new high-energy laser system to be installed at the ESRF, with increased energy (200 J) and
frequency doubling crystals.

\section{Photon-Matter interaction}
\subsection{Optical Laser}
Nearly all of the physical processes of long pulse laser-matter interactions are described by partial differential
equations that can be solved with careful construction of numerical simulation codes. Here, the laser-matter
interactions are modelled with the 1D radiation-hydrodynamics computer code ``Esther'' \cite{Colombier2005}
or the 2D code ``MULTI2D'' \cite{Ramis2009}.

Table~\ref{tab:esrf_long_pulse} summarizes the main optical pump laser
parameters for shock compression experiments at the ESRF.

For the 1D hydrocodes, a moving mesh is applied in an arbitrary Lagrangian-Eulerian framework where the coordinates of
the mesh contain the variables from which density can be calculated (mass of
each zone is fixed). The most important variables
considered in these codes are those associated with the extreme conditions generated by the high-power lasers: pressure (density),
temperature and velocity. Feedback of these hydrocode outputs are crucial in the design and implementation of laser shock/ramp
experiments with X-ray interactions.

The X-ray probe duration, during laser shock compression, can range from nanosecond down to femtosecond exposures.
X-ray pulses for shock compression studies at the ESRF are $\approx$ 100 ps
long. Table~\ref{tab:esrf_parameters} lists the most important optical laser
parameters.

It is crucial to have an understanding of shock transit times so that accurate timing of the X-ray probe,
with respect to the laser initialisation, can be made. The final density state reached by a shock, as calculated
from the hydrocode simulation packages, can define the shock velocity.

\subsubsection{Esther}
The Esther hydrocode was written by Patrick Combis and Laurent Videau of the CEA, Paris, France.
\footnote{The Esther hydrocode is currently presented in French only.}
A license to use Esther can be obtained by requesting access (via email) from the authors (for academic use only).
As part of the \texttt{simex\_platform}, a calculator interface is being developed to create the input files necessary for running
the hydrocode simulations. Esther output can be stored in a openPMD conform hdf5
file using a conversion tool which is part of \texttt{simex\_platform}.

\subsubsection{MULTI}
In the MULTI2D hydrocode \cite{Ramis2009},
the hydrodynamic equations solved by the code are combined with a multigroup method for radiation transport.

\subsection{X-ray matter interaction}
In the proposed experimental setup, discussed in detail further below, the shock compressed matter is diagnosed by
X-ray Absorption Fine Structure (XAFS) spectroscopy and radiography. Both techniques can be modeled with appropriate simulation tools as
described in the following.

In some cases the particle velocity of the interface between sample and window cannot be obtained directly as strong shocks can transform the transparent window to an opaque material. In those instances only the free surface velocity can be obtained by the velocity diagnostics and an iterative Lagrangian analysis (ILA) must be used to determine the sound speed; stress-density information can then be calculated. \textbf{(REFERENCE: S. D. Rothman and J. Maw, "Characteristics analysis of isentropic compression experiments (ice)", J. Phys. IV. France, 134, 745-750 (2006)).} Using the ILA, the velocity-time history during the compression can be obtained from the free surface measurements and from knowledge of the optical laser pulse to x-ray pulse delay, the exact state probed by the x-ray is obtained.  
\subsection{XAFS}
The XAFS signal reflects the near order of the shock compressed matter. The absorption edge is modulated by the ion-ion correlation function
$\Xi(\vec q)$.

\printbibliography
\end{document}


